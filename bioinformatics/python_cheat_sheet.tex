% Created 2024-01-24 Wed 10:50
% Intended LaTeX compiler: lualatex
\documentclass{assignments}
\usepackage{graphicx}
\usepackage{longtable}
\usepackage{wrapfig}
\usepackage{rotating}
\usepackage[normalem]{ulem}
\usepackage{amsmath}
\usepackage{amssymb}
\usepackage{capt-of}
\usepackage{hyperref}
\usepackage{minted}
\usepackage{physics}
\author{David Lewis}
\date{\today}
\title{}
\hypersetup{
 pdfauthor={David Lewis},
 pdftitle={},
 pdfkeywords={},
 pdfsubject={},
 pdfcreator={Emacs 30.0.50 (Org mode 9.7)}, 
 pdflang={English}}
\begin{document}

\section*{Assignments and Variables}
\label{sec:org7ffd298}
I you want to store information in a variable, you use the `=` operator.
\begin{minted}[fontsize=\scriptsize,breaklines=true,breakanywhere=true]{python}
# set X with 5
X = 5
print(X) # will print 5
\end{minted}

\begin{verbatim}
None
\end{verbatim}

\section*{Addition, subtraction, multiplication, division, exponents}
\label{sec:orge926527}
Math works, how cool! This follows classic order of operations as well.
\begin{minted}[fontsize=\scriptsize,breaklines=true,breakanywhere=true]{python}
print(5 + 5) # prints 10
print(1.1 + 5) # prints 6.1 (probably)
print(5 * 5) # prints 25
print(6 / 3) # prints 2
print(5**2) # prints 5^2 (25)
\end{minted}

\begin{verbatim}
None
\end{verbatim}

\section*{Logic}
\label{sec:org4de3066}
True and False! Boolean variables are true or false. The \texttt{and} keyword evaluates
to True if both the left and right are True, the \texttt{or} keyword evaluates to true if
either the left or right is True. The \texttt{!} and \texttt{not} mean the same thing, they invert
the output of the boolean (\texttt{not True} is the same as \texttt{False}).
\begin{minted}[fontsize=\scriptsize,breaklines=true,breakanywhere=true]{python}
X = True # X is a boolean in this case
print(5 == 5) # outputs True
print(5 == 3) # outputs False
print(True and False) # outputs False
print(True or False) # outputs True
print (not True) # outputs False
print (!True) # outputs False
\end{minted}

\section*{If}
\label{sec:org6e2745e}
If this, then that. The \texttt{if} keyword tests a logic function and runs some code
if it is true. \texttt{else} runs if the condition is false.
\#+begin\textsubscript{src} python
X = 5
if X == 5: \# if x is equal to 5

    print(``It's 5!'')
else: \# otherwise
    print(``It's not 5!'')
\#+end\textsubscript{src5}
\section*{Functions}
\label{sec:orgdbfb5da}
A function is a reusable bit of code. You can think of it as a subprogram, or a
small program that you can run inside the big program. They can take arguments
which can be used as variables inside the ``scope'' of the function.
\begin{minted}[fontsize=\scriptsize,breaklines=true,breakanywhere=true]{python}
def example_function(example_argument):
    # in the scope of the function
    print(example_argument)
# out of the scope of the function
# you cannot access "example_argument" out here
example_function("Hi There") # What do you think this will do?
\end{minted}

\begin{verbatim}
None
\end{verbatim}


Functions can also ``return'' values after processing them
\begin{minted}[fontsize=\scriptsize,breaklines=true,breakanywhere=true]{python}
def add(number1, number2):
    return number1 + number2
X = add(5, 6) # what do you think the value of X is?
print(X)
\end{minted}

\begin{verbatim}
None
\end{verbatim}

\section*{Loops}
\label{sec:org8ba1ab0}
Say you want to run the same code a bunch of times, but you don't want to write
it out every time or you don't know beforehand how many times you want it to
run. You need to use loops!
\subsection*{For loop}
\label{sec:orgcf54ae9}
A for loop runs a specific number of times and in python specifically, uses an
``iterable'' object as a way of determining how many times it runs. This could be
a range, list, dict, or set
\begin{minted}[fontsize=\scriptsize,breaklines=true,breakanywhere=true]{python}
# I want to print "help" 10 times
for i in range(10):
    print("help")
    print(i) # i will print 0-9 (ranges are a bit odd)
l = [1, 5, 6] # I want to print every member of this list
for member in l:
    print(member)
\end{minted}

\begin{verbatim}
None
\end{verbatim}

\subsection*{While loop}
\label{sec:org7438c1a}
While loops use a condition or boolean to determine when to stop, similar to the
way \texttt{if} works.
\begin{minted}[fontsize=\scriptsize,breaklines=true,breakanywhere=true]{python}
X = True
Y = 0
while X: # what do you think this will do?
    if Y < 10: # if Y is less then 10
        print(Y)
        Y = Y + 1
    else:
        X = False

\end{minted}

\begin{verbatim}
None
\end{verbatim}

\section*{Objects}
\label{sec:orgd43b6fa}
Almost everything in python is an ``object'', or an instance of a \texttt{class}. An easy
way to think about this is that you are an instance of class ``human''. You can
determine the class or type of an object using the \texttt{type} function.
\begin{minted}[fontsize=\scriptsize,breaklines=true,breakanywhere=true]{python}
X = "hi there"
print(type(X)) # prints <class str>
X = 5
print(type(X)) # prints <class int>
X = 5.1
print(type(X)) # prints <class float>
\end{minted}

\begin{verbatim}
None
\end{verbatim}


Each class has a different set of ``methods'' or functions associated with that
class. The \texttt{str} class has an \texttt{upper()} function which returns the string as all
upper case letters. This would not make sense for an integer, so there is no
upper function for integers. Classes can also define their own variables.
\begin{minted}[fontsize=\scriptsize,breaklines=true,breakanywhere=true]{python}
X = "hi There" # string
print(X.upper())
X = 5 # integer
print(X.upper()) # returns an error
\end{minted}

You can also make your own classes using the \texttt{class} keyword.
\begin{minted}[fontsize=\scriptsize,breaklines=true,breakanywhere=true]{python}
class Human:
    def __init__(self, name): # special function that creates the class, self is used to access class instance variables and functions
        self.name = name # creating a class variable called name and setting it to the given name

    def speak(self):
        print("My name is:", self.name) # You can access the name variable from a different function within the same class

human1 = Human("Mary") # runs the __init__ function
human2 = Human("Megan") # two instances of the same class human1 and human2

human1.speak() # runs the speak function (will print "Mary")
human2.speak() # runs the speak function (will print "Megan")
print(type(human1)) # what will this print?
\end{minted}

\begin{verbatim}
None
\end{verbatim}

\section*{Libraries}
\label{sec:org2e2cce8}
Say you wrote some code in a different file but you want to use it in your
current one. You can \texttt{import} other python files to achieve this goal. Python has
included a bunch of helpful libraries already, but more can be installed or
written yourself.
\begin{minted}[fontsize=\scriptsize,breaklines=true,breakanywhere=true]{python}
import math # built in library
# the math library has a lot of extra math stuff (straightforward right?)
print(math.sin(math.pi))

import numpy as np # needs to be installed (included with anaconda)
X = np.array([1, 2, 3]) # a vector has a magnitude and a direction
print(np.dot(X, X)) # performs the dot product
\end{minted}

\begin{verbatim}
None
\end{verbatim}
\end{document}