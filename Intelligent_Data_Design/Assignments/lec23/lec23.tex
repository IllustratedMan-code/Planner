% Created 2023-04-04 Tue 13:47
% Intended LaTeX compiler: lualatex
\documentclass[11pt]{article}
\usepackage[margin=0.5in]{geometry}
\usepackage{syntax}
\usepackage{pdfpages}
\usepackage{tcolorbox}
\usepackage{etoolbox}
\usepackage{environ}
\AtBeginEnvironment{quote}{\itshape}
\usepackage[ruled]{algorithm2e}
\let\oldtabular\tabular
\let\oldendtabular\endtabular
\NewEnviron{tabular2}[1]{\tcbox[left=0mm, right=0mm, top=0mm, bottom=0mm]{\oldtabular{#1}\BODY\oldendtabular}}
\BeforeBeginEnvironment{minted}{\begin{tcolorbox}}%
\AfterEndEnvironment{minted}{\end{tcolorbox}}
\BeforeBeginEnvironment{verbatim}{\begin{tcolorbox}}%
\AfterEndEnvironment{verbatim}{\end{tcolorbox}}
\usepackage{graphicx}
\usepackage{longtable}
\usepackage{wrapfig}
\usepackage{rotating}
\usepackage[normalem]{ulem}
\usepackage{amsmath}
\usepackage{amssymb}
\usepackage{capt-of}
\usepackage{hyperref}
\usepackage{minted}
\usepackage{physics}
\author{David Lewis}
\date{}
\title{Lecture 23}
\hypersetup{
 pdfauthor={David Lewis},
 pdftitle={Lecture 23},
 pdfkeywords={},
 pdfsubject={},
 pdfcreator={Emacs 30.0.50 (Org mode 9.6.1)}, 
 pdflang={English}}
\begin{document}

\maketitle
\section*{1.}
\label{sec:org31daf57}
\subsection*{a.}
\label{sec:org87e7ef5}
\begin{itemize}
\item PQR and XYZ are clustered
\item distance is low for PQR and distance is high for XYZ and vice versa
\end{itemize}
\subsection*{b.}
\label{sec:org4386e7a}
They are all somewhat similar to each other, but the range of attribute one for PQR
attributes 1 and 2 is \(\ge\) 15, but the range is <= 4 for attributes 3,4,5,6. So PQR
is more similar for attributes 3,4,5,6 than for attributes 1,2

For XYZ, the ranges are ordered by attributes smallest to greatest (most similar
to least similar) \(a_3 < a_1 = a_6 = a_5 < a_4\)

\subsection*{c.}
\label{sec:org9084c28}
PQRXYZ (no good sub clusters). There are no obvious clusterings.
\subsection*{d.}
\label{sec:orgd7d51d8}
a\textsubscript{4}, and a\textsubscript{5}
\subsection*{e.}
\label{sec:orge78519c}
b is the one we want to do subspace clustering on, because there are no good
clusters in the current dimensionality
\subsection*{f.}
\label{sec:org2a30b65}
\begin{itemize}
\item PQR (defining attributes a\textsubscript{2} and a\textsubscript{3} )
\item XYZ (defining attributes a\textsubscript{4} and a\textsubscript{5} )
\end{itemize}
\subsection*{g.}
\label{sec:org46f0091}
They are axis aligned, meaning that axis aligned algorithms are most effective
(clique, random-projections)
\subsection*{h.}
\label{sec:org7f34e9c}
Clique algorithm is vunerable to binning, meaning that some points will be mis-clustered
\subsection*{i.}
\label{sec:orgc6b1e89}
\begin{itemize}
\item Q and R are the randomly selected points
\item PQR is the subspace cluster (a\textsubscript{2} and a\textsubscript{3} are defining attributes)
\end{itemize}

\section*{2.}
\label{sec:org6e6317a}
\subsection*{a.}
\label{sec:org654f747}
linear combinations of existing dimensions, but not axis aligned
\subsection*{b.}
\label{sec:orge17eb0a}
The data is not axis aligned, meaning random projection cannot ``chunk'' data
effectively.
\subsection*{c.}
\label{sec:org43ecbfd}
Projective k-means, as it can handle linear cominations of existing dimensions
\subsection*{d.}
\label{sec:org48e6dba}
\begin{itemize}
\item choosing the number of clusters is hard
\item PCA is expensive and must be applied many times
\end{itemize}
\section*{3.}
\label{sec:org75e1fa2}
\subsection*{a.}
\label{sec:org6508239}
There are usually > 5 points in each cell (as far as I can tell, the graph is
hard to read)

\subsection*{b.}
\label{sec:org859cf11}
\begin{itemize}
\item estimate the number of cells that are filled change 0.1 increment dimension to
0.2 increments
\item 12 lines
\item Each line is 5 cells long
\item 60 cells would need to be evalutated
\end{itemize}
\end{document}