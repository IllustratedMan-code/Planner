% Created 2023-04-03 Mon 12:20
% Intended LaTeX compiler: lualatex
\documentclass[11pt]{article}
\usepackage[margin=0.5in]{geometry}
\usepackage{syntax}
\usepackage{pdfpages}
\usepackage{tcolorbox}
\usepackage{etoolbox}
\usepackage{environ}
\AtBeginEnvironment{quote}{\itshape}
\usepackage[ruled]{algorithm2e}
\let\oldtabular\tabular
\let\oldendtabular\endtabular
\NewEnviron{tabular2}[1]{\tcbox[left=0mm, right=0mm, top=0mm, bottom=0mm]{\oldtabular{#1}\BODY\oldendtabular}}
\BeforeBeginEnvironment{minted}{\begin{tcolorbox}}%
\AfterEndEnvironment{minted}{\end{tcolorbox}}
\BeforeBeginEnvironment{verbatim}{\begin{tcolorbox}}%
\AfterEndEnvironment{verbatim}{\end{tcolorbox}}
\usepackage{graphicx}
\usepackage{longtable}
\usepackage{wrapfig}
\usepackage{rotating}
\usepackage[normalem]{ulem}
\usepackage{amsmath}
\usepackage{amssymb}
\usepackage{capt-of}
\usepackage{hyperref}
\usepackage{minted}
\usepackage{physics}
\author{David Lewis}
\date{\today}
\title{}
\hypersetup{
 pdfauthor={David Lewis},
 pdftitle={},
 pdfkeywords={},
 pdfsubject={},
 pdfcreator={Emacs 30.0.50 (Org mode 9.6.1)}, 
 pdflang={English}}
\begin{document}

\section*{Multi dimensional CFG}
\label{sec:orgd42ca1f}
\begin{itemize}
\item The idea is that grammar can generate characters in multiple dimensions in parallel
\item Once a production rule is ``finished'', ie doesn't use the same symbol, the
extra dimenional characters will be ``folded'' into the one dimensional paradigm
\item I don't think this new grammar really fits the notion of a ``context free'' grammar,
however it is closest in syntax
\item I'll define the syntax using the following example
\end{itemize}
\subsection*{Example}
\label{sec:org20baae5}
\begin{itemize}
\item consider the language \(L = \{a^ib^ic^i: i \ge 0\}\)
\item This language \textbf{cannot} be generated using traditional context free grammars
\item we \emph{could} generate the language \(L_c = \{a^ib^i : i \ge 0\}\)
\item The resulting \(G_c\) would be: \(S \to aSb | \lambda\)
\item Context free grammars are limited by the number of parallel operations
(Left + Right=2)
\item We can define the ``number of parallel operations'' as degrees of freedom
\item If we allow the context free grammar more degrees of freedom we can generate a
2 dimensional projection of \(L\)
\end{itemize}

Consider the following grammar:
\setlength\arraycolsep{1pt}
\begin{tcolorbox}
\begin{equation}
\begin{matrix}&  b\\ S\to a & S & c & | & \lambda\end{matrix}
\end{equation}
\end{tcolorbox}

In the ``multi-dimensional'' grammar paradigm, the following derivation would be
valid:
\begin{tcolorbox}\begin{equation}
\begin{equation}
\begin{matrix}
      &   &   &   &    & b\\
      & b &   &   &    & b\\
 S\Rightarrow a & S & c & \Rightarrow & aa & S & cc \Rightarrow aabb\lambda cc \Rightarrow aabbcc  \end{matrix}
\end{equation}
\end{tcolorbox}

As you can see, this grammar generates the language \(L = \{a^ib^ic^i: i \ge
0\}\). This is possible because of what I'm going to call a ``fold''. When S is
substituted with a non-S symbol (\(\lambda\) in this case), the higher dimensions are projected down into
one dimension. Syntactically, I'm defining ``above'' as projecting to ``left'' and
vice versa.

The above example is showing the two dimensional case, but this concept should
be applicable to higher dimensions (not sure how useful they would be).

\subsection*{Advantages}
\label{sec:orge0b75be}
\begin{itemize}
\item I have a suspicion that this concept has feature parity with push down
automata (would love help proving!).
\item If that is the case, then this might be an easier way to define languages
accepted by push down automata
\end{itemize}
\subsection*{Limitations}
\label{sec:org959a4a2}
\begin{itemize}
\item I haven't done a lot of experimenting, but I don't think it is possible to
generate \(L = \{a^{i^2}: i\ge 0\}\)
\item I think you would need at least two symbols that generate in parallel, to
emulate two stacks in a turing machine
\end{itemize}
\end{document}