% Created 2023-03-21 Tue 17:31
% Intended LaTeX compiler: lualatex
\documentclass[11pt]{article}
\usepackage[margin=0.5in]{geometry}
\usepackage{syntax}
\usepackage{pdfpages}
\usepackage{tcolorbox}
\usepackage{etoolbox}
\usepackage{environ}
\usepackage[ruled]{algorithm2e}
\let\oldtabular\tabular
\let\oldendtabular\endtabular
\NewEnviron{tabular2}[1]{\tcbox[left=0mm, right=0mm, top=0mm, bottom=0mm]{\oldtabular{#1}\BODY\oldendtabular}}
\BeforeBeginEnvironment{minted}{\begin{tcolorbox}}%
\AfterEndEnvironment{minted}{\end{tcolorbox}}
\BeforeBeginEnvironment{verbatim}{\begin{tcolorbox}}%
\AfterEndEnvironment{verbatim}{\end{tcolorbox}}
\usepackage{graphicx}
\usepackage{longtable}
\usepackage{wrapfig}
\usepackage{rotating}
\usepackage[normalem]{ulem}
\usepackage{amsmath}
\usepackage{amssymb}
\usepackage{capt-of}
\usepackage{hyperref}
\usepackage{minted}
\usepackage{physics}
\author{David Lewis}
\date{\today}
\title{lec19}
\hypersetup{
 pdfauthor={David Lewis},
 pdftitle={lec19},
 pdfkeywords={},
 pdfsubject={},
 pdfcreator={Emacs 30.0.50 (Org mode 9.6.1)}, 
 pdflang={English}}
\begin{document}

\maketitle
\section*{1.}
\label{sec:orgb9f529e}
\begin{enumerate}
\item \(4^8 \cdot 3 = 96\)
\item It takes exponential time, permutations increase exponenetially with number
of data points
\end{enumerate}
\section*{2.}
\label{sec:org3a460d3}
\subsection*{a.}
\label{sec:orge603e2a}
\begin{minted}[fontsize=\scriptsize]{python}
D = [2, 3, 4, 8, 9, 10, 15, 16, 17]
M = [5, 14, 21]
A = []
for i in D:
    L = [(i-m)**2 for m in M]
    A.append(M[L.index(min(L))])
A
\end{minted}

\begin{center}
\begin{tabular2}{rrrrrrrrr}
5 & 5 & 5 & 5 & 5 & 14 & 14 & 14 & 14\\[0pt]
\end{tabular2}
\end{center}
\subsection*{b.}
\label{sec:org235e1c9}
\begin{minted}[fontsize=\scriptsize]{python}
import random
def update_centroids(M, A):
    def reassign_empty(M, A):
        for i in M:
            if i not in A:
                new = random.choice(D)
                A[D.index(new)] = i
                return reassign_empty(M, A)
        return A
    A = reassign_empty(M, A)
    M_new = [0 for m in M]
    S_len = [0 for m in M]
    for d,a in zip(D, A):
        for m in range(len(M)):
            if a == M[m]:
                S_len[m] += 1
                M_new[m] += d
    M = [x/s for (x, s) in zip(M_new, S_len)]
    return M
M = update_centroids(M, A)
M
\end{minted}

\begin{center}
\begin{tabular2}{rrr}
4.5 & 14.5 & 8.0\\[0pt]
\end{tabular2}
\end{center}

\subsection*{c.}
\label{sec:org904bdba}
\begin{minted}[fontsize=\scriptsize]{python}
D = [2, 3, 4, 8, 9, 10, 15, 16, 17]
A = []
for i in D:
    L = [(i-m)**2 for m in M]
    A.append(M[L.index(min(L))])
A
\end{minted}

\begin{center}
\begin{tabular2}{rrrrrrrrr}
4.5 & 4.5 & 4.5 & 8.0 & 8.0 & 8.0 & 14.5 & 14.5 & 14.5\\[0pt]
\end{tabular2}
\end{center}
\subsection*{d.}
\label{sec:org65f6d76}
In my code above, I select a random member of D to be reassigned to the empty
centroid. I made this function recursive in case the reasignment creates a new
empty centroid. The function will run until each centroid has at least one assignment.
\subsection*{e.}
\label{sec:org1ba31bb}
\begin{verbatim}
Assignment: assign each observation to the closest cluster
while assignments change:
    while centroid has no assignments:
        assign random observation to empty centroid
    recalculate centroids:
        m_i = (1/length(assignments to centroid)) * sum(each assignment to centroid)
\end{verbatim}

Time complexity: \(O(n^{dk+1}\log n)\)

\section*{3.}
\label{sec:org4865c95}
\subsection*{a.}
\label{sec:orgd745f99}
\begin{minted}[fontsize=\scriptsize]{python}
from scipy.stats import norm
D = [1, 4, 6, 9]
Mu = [1, 9]
Sigma = [1, 1]
results = []
for mu, sigma in zip(Mu, Sigma):
    results.append([f"mu={mu}, sigma={sigma}"]+ list(norm.pdf(D, mu, sigma)))
results
\end{minted}

\begin{center}
\begin{tabular2}{lrrrr}
mu=1, sigma=1 & 0.3989422804014327 & 0.0044318484119380075 & 1.4867195147342979e-06 & 5.052271083536893e-15\\[0pt]
mu=9, sigma=1 & 5.052271083536893e-15 & 1.4867195147342979e-06 & 0.0044318484119380075 & 0.3989422804014327\\[0pt]
\end{tabular2}
\end{center}

\subsection*{b.}
\label{sec:orgbceb426}
\begin{minted}[fontsize=\scriptsize]{python}
D = list(zip(results[0][1:], results[1][1:]))
posterior = []
for i1, i2 in D:
    normalization = i1 + i2
    posterior.append([i1/(normalization), i2/(normalization)])

posterior
\end{minted}

\begin{center}
\begin{tabular2}{rr}
0.9999999999999873 & 1.2664165549094016e-14\\[0pt]
0.9996646498695335 & 0.0003353501304664781\\[0pt]
0.0003353501304664781 & 0.9996646498695335\\[0pt]
1.2664165549094016e-14 & 0.9999999999999873\\[0pt]
\end{tabular2}
\end{center}
\subsection*{c.}
\label{sec:org94e46ac}
\begin{minted}[fontsize=\scriptsize]{python}
prior = [0, 0]
for i in posterior:
    prior[0] += i[0]
    prior[1] += i[1]
prior = [i/len(posterior) for i in prior]
prior
\end{minted}

\begin{center}
\begin{tabular2}{rr}
0.5 & 0.5\\[0pt]
\end{tabular2}
\end{center}
\subsection*{d.}
\label{sec:org02df948}
\begin{minted}[fontsize=\scriptsize]{python}
mu = [0, 0]
mu_d = [0, 0]
for p, (d1, d2) in zip(posterior, D):
    mu[0] += p[0] * d1
    mu[1] += p[1] * d2
    mu_d[0] += p[0]
    mu_d[1] += p[1]
mu[0] = mu[0] / mu_d[0]
mu[1] = mu[1] / mu_d[1]
mu
sigma = [0, 0]
sigma_d = [0, 0]
for p, (d1, d2) in zip(posterior, D):
    sigma[0] += p[0] * (d1 - mu[0])**2
    sigma[1] += p[1] * (d2 - mu[1])**2
    sigma_d[0] += p[0]
    sigma_d[1] += p[1]
sigma[0] = sigma[0] / sigma_d[0]
sigma[1] = sigma[1] / sigma_d[1]
sigma
[["mu_1", "mu2"], mu, ["sigma2_1", "sigma2_2"], sigma]
\end{minted}

\begin{center}
\begin{tabular2}{rr}
mu\textsubscript{1} & mu2\\[0pt]
0.20168632154549704 & 0.20168632154549704\\[0pt]
sigma2\textsubscript{1} & sigma2\textsubscript{2}\\[0pt]
0.03890991659421365 & 0.03890991659421365\\[0pt]
\end{tabular2}
\end{center}
\subsection*{e}
\label{sec:orgf8da31d}
a and b are part of the E step, c and d are part of the m step
\end{document}