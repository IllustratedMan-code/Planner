% Created 2023-01-22 Sun 13:47
% Intended LaTeX compiler: lualatex
\documentclass[11pt]{article}
\usepackage{graphicx}
\usepackage{longtable}
\usepackage{wrapfig}
\usepackage{rotating}
\usepackage[normalem]{ulem}
\usepackage{amsmath}
\usepackage{amssymb}
\usepackage{capt-of}
\usepackage{hyperref}
\usepackage[margin=0.5in]{geometry}
\author{David Lewis}
\date{1/22/2023}
\title{Homework 1}
\hypersetup{
 pdfauthor={David Lewis},
 pdftitle={Homework 1},
 pdfkeywords={},
 pdfsubject={},
 pdfcreator={Emacs 28.2 (Org mode 9.6)}, 
 pdflang={English}}
\begin{document}

\maketitle
\section*{1.}
\label{sec:orgc364e5d}
\begin{enumerate}
\item (False) Cable based networks send signals along shared cables, but DSL does not.
\item (False) Fiber leaving the central office is shared, goes into splitter which
has a dedicated fiber for each home (about 100).
\item (False) Many ISPs can only be reached through a tier 1 ISP, so content
providers must pay tier 1 ISPs to access them.
\end{enumerate}

\section*{2.}
\label{sec:org98b616d}
\subsection*{a.}
\label{sec:orgceb741f}
16, consider if switch A made 4 connections with switch B, B made 4 connections
with C, C made 4 connections with D, and D made 4 connections with A.
\subsection*{b.}
\label{sec:org1f03852}
8, 4 connections between A and B or A and D, 4 connections between B and C or D
and C.
\subsection*{c.}
\label{sec:org41833c4}
yes, route 2 connections from A to C through B and 2 through D, route 2
connections from B to D through A and 2 through C.
\section*{3}
\label{sec:orgdcfa2b0}
\subsection*{a.}
\label{sec:orgf653fc7}
When using store-and-forward packet switching, the entire packet must arrive at
each switch before it can be sent to next one, since the message is not being
split into multiple packets, the entire message must arrive before it can be
sent to the next link. There are 2 links in the chain, so 3 ``hops'' must occur.

Depending on whether Megabit s or binary megabits (mebibits) are used, the time
required will either be \((7.5 \cdot 10^6 / 1 \cdot 10^6) \cdot 3 \text{hops} = 22.5 \text{s}\) or \((7.5 \cdot 10^6
/ 1024^2) \cdot 2 \text{hops}  = 21.45 \text{s}\) respectively.
\subsection*{b.}
\label{sec:org41a3917}
\begin{itemize}
\item Decimal megabits will be used for this question
\item Each packet is \(\frac{7.5 \cdot 10^6}{2500} = 3000\text{bits/packet}\).
\item First packet arrives in \((3000bits/1 \cdot 10^6) \cdot 3 \text{hops} = 0.009 \text{s}\)
\item After the first packet arrives, each packet will arrive in 0.003 seconds
\item Total time is \(0.009\text{s} + 2500 \cdot 0.003 = 7.509 \text{s}\)
\end{itemize}

\section*{4}
\label{sec:org394f764}
\subsection*{a.}
\label{sec:org7bfdf1a}
\begin{itemize}
\item \(d_{prop} = \frac{20000000 \text{m}}{2.5 \cdot 10^8 \text{m/s}} = 0.08 \text{s}\)
\item \(R \cdot d_{prop} = 2 \text{Mb/s} \cdot 0.08 \text{s} = 0.16 \text{Mb}\)
\end{itemize}
\subsection*{b.}
\label{sec:org77cae34}
The maximum number of bits that will be in the link is the same as the
bandwidth-delay product, so \(0.16 \text{Mb} = 0.16 \cdot 10^6 \text{bits}\) or
\(0.16 \cdot 1024^2 = 167772 \text{bits}\)  depending if you are using the new-ish
decimal megabit or binary megabit(mebibit)

\subsection*{c.}
\label{sec:orgc758d0e}
The bandwith delay product is the time it takes to propagate through the
link for every bit multiplied by the number of bits that can travel
through the link per time. This is essentially equivalent to filling up the
link until it can take no more bits, then counting all of the bits.
\subsection*{d.}
\label{sec:org0682be8}
Depending on whether one is using megabits or mebibits, the length of a bit in
the link is \(\frac{20000000 \text{m}}{1.6 \cdot 10^5\text{bits}} = 125
\text{m/bit}\) or \(\frac{20000000 \text{m}}{167772\text{bits}} = 119.2
\text{m/bit}\). A football field is approxamately 91.44 meters, so the bit is
longer in either case.
\subsection*{e.}
\label{sec:org0d3afbf}
\begin{itemize}
\item \(\text{bit length} = \frac{\text{m}}{R \cdot \frac{\text{m}}{\text{s}}} = \frac{s}{R}\)
\end{itemize}

\section*{5}
\label{sec:orge041000}
\begin{itemize}
\item One advantage is the ability to change the implementation of a service without
affecting the other components. This is really useful for determining the
cause of and fixing a problem without affecting the rest of the system. It
also allows for more flexibility, as the way an operation occurs could be
changed (copper to fiber).
\item Another advantage of layering is the structure it provides. Layering is
conducive to discussion of a well-defined, specific part of a large and
complex system, making communication and collaboration much simpler. This
reduces the amount of prerequisite knowledge someone might need to contribute
to a project (I wouldn't need to know how the entire system worked, only
the part I was responsible for).
\item One disadvantage is that one layer may duplicate a lower-layer functionality.
This might mean that error recovery happens twice (double the processing time)
without any benefit.
\item Another disadvantage is that functionality on one layer may need information
from another layer, which prevents layers from being separate. Maybe two
layers need the current timestamp to function correctly, which if fixed, would
destroy or complicate the layer structure.
\end{itemize}
\end{document}
