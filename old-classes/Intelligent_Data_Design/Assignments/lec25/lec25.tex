% Created 2023-04-11 Tue 13:54
% Intended LaTeX compiler: lualatex
\documentclass[11pt]{article}
\usepackage[margin=0.5in]{geometry}
\usepackage{syntax}
\usepackage{pdfpages}
\usepackage[most]{tcolorbox}
\usepackage{etoolbox}
\usepackage{environ}
\AtBeginEnvironment{quote}{\itshape}
\usepackage[ruled]{algorithm2e}
\let\oldtabular\tabular
\let\oldendtabular\endtabular
\NewEnviron{tabular2}[1]{\tcbox[left=0mm, right=0mm, top=0mm, bottom=0mm]{\oldtabular{#1}\BODY\oldendtabular}}
\BeforeBeginEnvironment{minted}{\begin{tcolorbox}[enhanced, breakable, skin first=enhanced, skin middle=enhanced, skin last=enhanced]}%
\AfterEndEnvironment{minted}{\end{tcolorbox}}
\BeforeBeginEnvironment{verbatim}{\begin{tcolorbox}}%
\AfterEndEnvironment{verbatim}{\end{tcolorbox}}
\usepackage{graphicx}
\usepackage{longtable}
\usepackage{wrapfig}
\usepackage{rotating}
\usepackage[normalem]{ulem}
\usepackage{amsmath}
\usepackage{amssymb}
\usepackage{capt-of}
\usepackage{hyperref}
\usepackage[breaklines]{minted}
\usepackage{physics}
\author{David Lewis}
\date{}
\title{Lecture 25}
\hypersetup{
 pdfauthor={David Lewis},
 pdftitle={Lecture 25},
 pdfkeywords={},
 pdfsubject={},
 pdfcreator={Emacs 30.0.50 (Org mode 9.6.1)}, 
 pdflang={English}}
\begin{document}

\maketitle

\section*{1.}
\label{sec:org2a7b3b0}
\subsection*{a.}
\label{sec:org9487f7f}
It looks as if The coefficient of A2 is around 2, while the coefficient of A1 is
around 1, set up in the equation of the form \(1A_1 + 2A_2 = Y\)
\subsection*{b.}
\label{sec:org7913dc7}
\begin{itemize}
\item \((D^TD)^{-1}D^Ty\)
\end{itemize}
\begin{minted}[fontsize=\scriptsize,breaklines=true,breakanywhere=true]{python}
import numpy as np
D = [[1, 2, 1],
     [2, 2, 1],
     [3, 4, 1],
     [1, 4, 1]]
Y = [5.1, 6.3, 10.8, 8.8]
DT = np.transpose(D)
DI = np.linalg.inv(np.matmul(DT, D))
W = np.matmul(np.matmul(DI, DT), Y)
W1 = W
w = W[:-1]
b = W[-1]
f"w={w}, b={b}"
\end{minted}

\begin{verbatim}
w=[1.04 1.79], b=0.5600000000000058
\end{verbatim}


\begin{minted}[fontsize=\scriptsize,breaklines=true,breakanywhere=true]{python}
def SSE(D):
    yhat = np.sum(D * W, axis=1)
    return sum((Y - yhat)**2)
SSE(D)
\end{minted}

\begin{verbatim}
0.016000000000000028
\end{verbatim}

\subsection*{c}
\label{sec:orgf089031}
\begin{itemize}
\item \(w = (X^TX + \alpha I)^{-1}X^Ty\)
\end{itemize}
\begin{minted}[fontsize=\scriptsize,breaklines=true,breakanywhere=true]{python}
alpha = 1
DT = np.transpose(D)
DI = np.linalg.inv(np.matmul(DT, D) + alpha * np.identity(3))
W = np.matmul(np.matmul(DI, DT), Y)
W2 = W
w = W[:-1]
b = W[-1]
f"w={w}, b={b}"
\end{minted}

\begin{verbatim}
w=[1.02345679 1.76049383], b=0.5419753086419743
\end{verbatim}

\begin{minted}[fontsize=\scriptsize,breaklines=true,breakanywhere=true]{python}
SSE(D)
\end{minted}

\begin{verbatim}
0.09464563328761147
\end{verbatim}
\subsection*{d.}
\label{sec:orga985df5}
\begin{minted}[fontsize=\scriptsize,breaklines=true,breakanywhere=true]{python}
p = [2, 5, 1]
[["b", np.dot(p, W1)], ["c", np.dot(p, W2)]]

\end{minted}

\begin{center}
\begin{tabular2}{lr}
b & 11.590000000000016\\[0pt]
c & 11.391358024691357\\[0pt]
\end{tabular2}
\end{center}
\subsection*{e.}
\label{sec:orgb7ab1f7}
The SSE is smaller than the linear regression model, this indicates that the
model has less error
\section*{2.}
\label{sec:orgf818dc7}
\subsection*{a.}
\label{sec:org5c3d16d}
\begin{itemize}
\item we'll rewrite y as \(y = w^Tx = Xw\)
\item X is matrix made of x rows
\item then we can say \(J(w) = \|y-\hat y\|^2 = (y-\hat y)^T(y - \hat y)\)
\item \(J(w) = (Xw-\hat y)^T(Xw - \hat y)\)
\item \(\frac{dJ}{dw} = 2X^TXw - 2X^Ty = 0\)
\item \(X^TXw = X^Ty\)
\item \(w = (X^TX)^{-1}X^Ty\)
\end{itemize}
\subsection*{b.}
\label{sec:org92a3147}
logistic regression can classify a datset by determining the probablility of
discrete values. If the discrete values are binary, then we can do binary
classification this way. The logistic regression is ideal for this because the
logistic function ``clamps'' to specific ends of the distribution (most points are
given a plus or minus value).
\end{document}