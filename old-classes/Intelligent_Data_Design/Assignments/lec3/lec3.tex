% Created 2023-01-17 Tue 15:23
% Intended LaTeX compiler: lualatex
\documentclass[11pt]{article}
\usepackage{graphicx}
\usepackage{longtable}
\usepackage{wrapfig}
\usepackage{rotating}
\usepackage[normalem]{ulem}
\usepackage{amsmath}
\usepackage{amssymb}
\usepackage{capt-of}
\usepackage{hyperref}
\usepackage{minted}
\usepackage[margin=0.5in]{geometry}
\author{David Lewis}
\date{1/17/2022}
\title{Lec 3: Classroom activity}
\hypersetup{
 pdfauthor={David Lewis},
 pdftitle={Lec 3: Classroom activity},
 pdfkeywords={},
 pdfsubject={},
 pdfcreator={Emacs 28.2 (Org mode 9.6)}, 
 pdflang={English}}
\begin{document}

\maketitle
\section*{1.}
\label{sec:org1b8e936}
Matrices need to be symmetric \textbf{and} positive semi-definite.
\subsection*{a.}
\label{sec:org54d6a9b}
\begin{itemize}
\item \(\begin{pmatrix}2 && 3 \\ 3 && 3 \end{pmatrix}^T = \begin{pmatrix}2 && 3 \\ 3 && 3 \end{pmatrix}\)
\item This matrix is symmetric
\item \(\begin{vmatrix}2 && 3 \\3 && 3\end{vmatrix} = 2 \times 3 - 3 \times 3 < 0\)
\item Determinante is less than 0, so matrix is not positive semi-definite
\item Matrix is not symmetric \textbf{and} semi-definite, therefore it is not a covariance matrix
\end{itemize}
\subsection*{b.}
\label{sec:orge5ad79d}
\begin{itemize}
\item \(\begin{pmatrix}7 && 6 \\ 5 && 7 \end{pmatrix}^T = \begin{pmatrix}7 && 5 \\ 6 && 7 \end{pmatrix}\)
\item This matrix is not symmetric
\item Matrix is not symmetric \textbf{and} semi-definite, therefore it is not a covariance matrix
\end{itemize}
\section*{2.}
\label{sec:orgbae3692}
\begin{itemize}
\item Distance = \(\sqrt{(x_1-y_1)^2 + (x_2-y_2)^2}\)
\end{itemize}
\begin{minted}[fontsize=\scriptsize]{python}
import math
def dist(x, y):
    return math.sqrt(((x[0]-y[0])**2) + ((x[1]-y[1])**2))
u = (2, 3)
results = {}
results["2, 5"] = dist(u, (2,5))
results["-5, 3"] = dist(u, (-5,3))
results["-2, 0"] = dist(u, (-2,0))

return list(results.items())
\end{minted}


\begin{center}
\begin{tabular}{lrr}
point & distance from mean & Rank (decreasing probability density)\\
\hline
2, 5 & 2.0 & 1\\
-2, 0 & 5.0 & 2\\
-5, 3 & 7.0 & 3\\
\end{tabular}
\end{center}
\section*{3.}
\label{sec:org10f5d78}
\begin{itemize}
\item 4 variables: p, q, r, s
\item \(\begin{pmatrix} \sigma_{p}^2 && \sigma_{pq} && \sigma_{pr} && \sigma_{ps} \\ \sigma_{qp}&& \sigma_{q}^2 &&
  \sigma_{qr} && \sigma_{qs} \\  \sigma_{rp} && \sigma_{rq} && \sigma_{r}^2 && \sigma_{rs}  \\  \sigma_{sp} && \sigma_{sq} && \sigma_{sr} && \sigma_{s}^2\end{pmatrix}\)
\item Covariance is defined as: \(\sigma_{xy} = \frac{\sum(x_i-\mu_x)(y_i-\mu_y)}{n}\)
\end{itemize}
\section*{4.}
\label{sec:org1cc9540}
\subsection*{1.}
\label{sec:org7b523ab}
Diagram 1 has a positive covariance (up from left to right) so the appropriate
covariance matrix is: \(\begin{bmatrix}8 && 7 \\ 7 && 7\end{bmatrix}\)
\subsection*{2.}
\label{sec:org8d25f31}
Diagram 2 has a negative covariance (down from left to right) so the appropriate
covariance matrix is:
\(\begin{bmatrix}7 && -5 \\
                -5 && 6 \end{bmatrix}\)
\subsection*{3.}
\label{sec:org16d64bd}
The covariance is 0 (no slope in distribution) and the points are distributed more
along the x axis than the y axis, meaning the x variance is higher than
the y variance. The appropriate matrix is therefore:
\(\begin{bmatrix} 2 && 0 \\ 0 && 1 \end{bmatrix}\)
\subsection*{4.}
\label{sec:orgf35e578}
This \emph{could} be solved by process of elimination, but the covariance is 0 (no
slope in distribution), and the points are distributed more
along the y axis than along the x axis, meaning that the y variance is higher than
the x variance. The appropriate matrix is therefore:
\(\begin{bmatrix}1 && 0 \\ 0 && 4 \end{bmatrix}\)
\section*{5.}
\label{sec:orgf517ce1}
\subsection*{a.}
\label{sec:org85344b0}
Red is the mode (the most common value)
\subsection*{b.}
\label{sec:orge03ec8a}
\begin{center}
\begin{tabular}{l|rrr|r}
 & Red & Black & White & Total\\
\hline
Nissan & 2 & 0 & 0 & 2\\
Toyota & 1 & 1 & 1 & 3\\
Honda & 1 & 1 & 2 & 4\\
\hline
Total & 4 & 2 & 3 & 9\\
\end{tabular}
\end{center}
\subsection*{c.}
\label{sec:org9a552b2}
\begin{itemize}
\item \(\rho(\text{Manufacturer}=\text{honda}) = \frac{4}{9}\)
\item \(\rho(\text{Color}=\text{white}) = \frac{3}{9} = \frac{1}{3}\)
\end{itemize}
\subsection*{d.}
\label{sec:orgc879e23}
\begin{itemize}
\item Expected number of points (ENP) = total * probability if variables are
independent (which is 1/3 in this case)
\item ENP(Manufacturer = Honda) = \(9 \cdot \frac{1}{3} = 3\)
\item ENP(Color=White) = \(9 \cdot \frac{1}{3} = 3\)
\end{itemize}
\subsection*{e.}
\label{sec:orgc2171d5}
\begin{itemize}
\item degrees of freedom = q = \((m_1-1)(m_0-1) = (3-1)^2 = 4\)
\end{itemize}
\end{document}
